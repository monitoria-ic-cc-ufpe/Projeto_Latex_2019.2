\documentclass[a4paper, 10pt]{article}
\usepackage[utf8]{inputenc}
\usepackage[portuguese]{babel}


\title{Empreendimentos em Informática}
\author{Amanda Alves Guimarães}
\date{October 2019}

\usepackage{natbib}
\usepackage{graphicx}
\setlength{\parindent}{10pt}
\usepackage{indentfirst}

\begin{document}

\maketitle

\section{Introdução}
Empreendimentos em Informática (IF781) é uma disciplina eletiva oferecida pelo Centro de Informática da UFPE aos alunos dos cursos de Ciência da Computação e Engenharia da Computação. Ela possui carga horária total de 75 horas e não exige pré-requisitos aos alunos que pretendem cursá-la. \par
No semestre eletivo atual (2019.2), Paulo Adeodato é o professor que ministra essa disciplina. As aulas possuem duração de duas horas e são ofertadas no período da manhã. Nas terças-feiras, começam às 10h e, nas quintas-feiras, começam às 8h. A disciplina é ministrada na sala E423.

\section{Relevância}
Durante os últimos 100 anos, a ciência e a tecnologia se desenvolveram de maneira significativa. A criação dos computadores foi um marco importante na história da humanidade, pois foi um dos fatores que deram início à Terceira Revolução Industrial. Dessa forma, a maioria das empresas começou a funcionar de maneira diferente: adaptaram-se às novas tecnologias devido ao custo-benefício. Consequentemente, a demanda a essas tecnologias aumentou e novas empresas foram criadas para suprir essa procura. Um exemplo bastante conhecido de uma empresa nas áreas da Informática e da Computação é a IBM, fundada em 1911, pela iniciativa do empresário Charles Flint.\cite{ibm}

Nos dias atuais, o número de empresas na área de Informática é bastante alto. Muitas vezes, novas empresas desse ramo não conseguem um lugar satisfatório no mercado porque não possuem um diferencial, ou seja, uma característica marcante que faça com que elas se destaquem dos outros empreendimentos. Por isso, é de extrema importância que um profissional da área de tecnologia da informação que tem interesse em criar uma empresa tenha conhecimento no ramo do empreendedorismo.

Vale ainda ressaltar que, durante os últimos dez anos, Recife vem gerando cada vez mais empresas na área da tecnologia da informação. Iniciativas como o Porto Digital, que conta com 250 empresas embarcadas, contribuem bastante para esse crescimento. Dessa forma, é essencial que os alunos do Centro de Informática que pretendem empreender tenham contato, durante a formação, com o empreendedorismo na área da Informática.

\begin{figure}[h]
    \centering
    \includegraphics[width=8cm]{portodigital.jpg}
    \caption{Porto Digital, Recife.\cite{portodigital}}

    \label{fig:my_label}
\end{figure}

\section{Relação com outras disciplinas}
A disciplina de Empreendimentos em Informática não possui pré-requisitos ou co-requisitos. Entretanto, faz parte do perfil empreendedor. Dessa forma, ela complementa o conhecimento adquirido em outras disciplinas, como Gestão de Negócios, Planejamento e Gerenciamento de Negócios, Economia e Mercado para o Empreendedor e Contabilidade Financeira e Gerencial. \cite{if781}


\bibliographystyle{abntex2-num}
\bibliography{aag.bib}

\end{document}
