{\documentclass[a4paper, 10pt]{article}
\usepackage[utf8]{inputenc}

\title{IF743 - Segurança de Sistemas}
\author{José Wilson Cavalcante Ferreira Junior}
\date{November 2019}


\usepackage{natbib}
\usepackage{graphicx}
\setlength{\parindent}{10pt}
\begin{document}

\maketitle

\section{Introdução}
É notável nos dias de hoje a grande quantidade de pessoas que usam a internet e o quanto que elas acabam precisando dela para o decorrer do seu dia a dia, esse fato não é muito diferente de empresas as quais muitas vezes, apesar de possuirem uma rede de computadores interligados, ações como pagamentos, licitações, transações bancárias necessitam de acesso à internet.Existe porém uma ameaça na internet que são os conhecidos ataques ou invasões que podem acabar num grande prejuízo para a empresa tanto monetário quanto ter seu nome manchado no mercado.\cite{intro} 



\begin{figure}[h!]
\centering
\includegraphics[scale=0.4]{segurancadesistemasic.jpg}
\caption{Segurança de Sistemas}
\citep{imagemIntro}
\label{fig:universe}
\end{figure}

\section{Criptografia e Firewall}
Uma ferramenta muito utilizada na segurança dos dados do sistema , a qual consistem em codificar e decodificar os dados que se deseja proteger que podem ser um em específico como uma senha ou todos os dados de um banco de dados, a leitura desses dados uma vez criptografados é possivel pelo acesso à chave de criptografia.\cite{cripto}
Uma outra ferramenta de segurança é o Firewall usado para segurança de redes o qual monitora o que entra e sai, são concedidas permissões e bloqueios são efetuados em função de regras para assim garantir a segurança de uma rede.\cite{firewall}

\begin{figure}[h!]
\centering
\includegraphics[scale=0.3]{criptografia.jpg}
\caption{Criptografia}
\citep{imagemCripto}
\end{figure}


\bibliographystyle{plain}
\bibliography{jwcfj}
\end{document}
}
