\documentclass[10pt]{article}
\usepackage[utf8]{inputenc}
\usepackage[portuguese]{babel}

\title{IF803 - Introdução a Biologia Molecular}
\author{Jhenne Cruz}
\date{Novembro 2019}

\usepackage{natbib}
\usepackage{graphicx}

\begin{document}

\maketitle

\section{Introdução}

\qquad O objetivo principal da disciplina em questão é apresentar a área de Biologia Molecular Computacional, introduzindo aos alunos conceitos essenciais da Biologia para que a partir do aprendizado dos fundamentos, esses sejam empregados na área de aplicação, dos problemas práticos de Bio-Informática e Biologia Computacional que envolve a manipulação e análise de dados biológicos e problemas da área, juntamente com abordagens computacionais para a sua solução.\citep{intro}

Em relação a grade curricular, a disciplina começa introduzindo uma visão geral da Biologia Molecular, em seguida aborda conceitos genéticos relacionados ao DNA relacionando-os com estrutura computacional. As formas de avaliação são através de projetos e testes ao longo do semestre.\cite{intro}


\begin{figure}[h!]
\centering
\includegraphics[scale=0.25]{imagem.jpg}
\caption{Imagem Biologia molecular computacional}\cite{imagem}
\label{fig:imagem}
\end{figure}

\section{Relevância}
\qquad O curso de Ciência da computação remete a ideia do conhecimento das ciências exatas e, sobretudo, tecnologia. Entretanto, a disciplina de Biologia Molecular Computacional abrange outras áreas de conhecimento, como a génetica, o que amplia o conhecimento dos alunos. Outrossim, é que a disciplina possui continuidade, ou seja, caso o aluno queira aprofundar-se nessa área é possível seguir estudando, pois há outras cadeiras eletivas nessa área, assim como programas de mestrado. Como é uma disciplina que envolve conhecimentos multidisciplinares, o aluno terá que se dedicar mais tempo, pois além da questão computacional e de programação, é necessário que haja conhecimento no campo da biologia para assim poder resolver os problemas propostos . 

No que concerne a importância para a sociedade, a biologia computacional colabora para a decodificação das informações contidas nos genes e de como elas agem fisiologicamente, influenciando processos como memória, elasticidade da pele, mutações etc.\cite{importancia}


\section{Relação com outras disciplinas}


\vspace{0.3cm}
\begin{tabular}{|p{3.0cm}|p{7.0cm}|}
\hline
Disciplinas & Relação\\
\hline


IF806 - Tópicos Avançados em Bio Informática &
Compartilha com IF803 o tópico de Bancos de Dados Biológicos, os quais servem para guardar seqüências de ácidos nucléicos e aminoácidos e suas respectivas anotações.\cite{IF806}\\
\hline 
IF804 - Comparação e Análise de Seqüências de DNA  & 
Ambas possuem  assuntos gerais relacionados com o DNA e a parte computacional,como, por exemplo: Introdução a Biologia Molecular Computacional. \cite{IF804}\\
\hline 





\end{tabular}

\bibliographystyle{plain}
\bibliography{jdac2}
\end{document}
