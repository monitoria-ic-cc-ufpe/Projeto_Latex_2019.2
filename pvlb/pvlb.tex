\documentclass[a4paper, 10pt]{article}
\usepackage[utf8]{inputenc}
\usepackage[portuguese]{babel}

\title{Engenharia Software e Sistemas}
\author{Pedro Victor Lima Barbosa }
\date{November 2019}

\usepackage{graphicx}
\usepackage{natbib}
\usepackage{indentfirst}
\setlength{\parindent}{10pt}

\begin{document}

\maketitle

\section{Introdução}
Engenharia de software é uma área da computação voltada à especificação, desenvolvimento, manutenção e criação de software, com a aplicação de tecnologias e práticas de gerência de projetos e outras disciplinas, visando organização, produtividade e qualidade.\citep{wikibooks}

\section{Relevância}
Sua relevância é a capacidade de ensinar os estudantes de ciência da computação a dar um tratamento mais sistemático e controlado ao desenvolvimento de sistemas de software complexos. Um sistema de software complexo se caracteriza por um conjunto de componentes abstratos de software (estruturas de dados e algoritmos) encapsulados na forma de procedimentos, funções, módulos, objetos ou agentes interconectados entre si.\citep{if682}

A ementa oficial da disciplina é a seguinte:

•	Planejamento e Gerenciamento de Software

•	Requisitos de Software

•	Análise e Projeto de Software

•	Codificação de Software

•	Depuração e Testes \citep{if682}

\begin{figure}
    \centering
    \includegraphics[width=8cm]{imagemSoftware.jpg}
    \caption{Programação, Engenheiro de Software}
    \label{fig:my_label}
\end{figure}

\section{Relação com outras disciplinas}
A cadeira Engenharia de Software e Sistemas se relaciona com as cadeiras: Algoritmos e Estruturas de Dados, e Logica para Computação. O objetivo principal deste curso é estudar, analisar, discutir, e aplicar conceitos de Engenharia de Software. 

Portanto, não se trata apenas de programação, uma atividade que pode ser desenvolvida de forma independente de outras pessoas, mas de atividades que requerem trabalho em equipe e capacidade de comunicação. \citep{cinWiki}

\bibliographystyle{ieeetr}
\bibliography{pvlb}


\end{document}
