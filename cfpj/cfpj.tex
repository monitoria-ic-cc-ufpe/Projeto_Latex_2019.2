\documentclass{article}
\usepackage[utf8]{inputenc}

\title{Gerenciamento de Dados e Informação}
\author{Carlos Frederico Pereira Júnior }
\date{Novembro 2019}

\usepackage{natbib}
\usepackage{graphicx}

\begin{document}

\maketitle

\section{Introdução}
Gerenciamento de Dados e Informação é uma disciplina obrigatória do quarto período para os estudantes dos cursos de Ciência da Computação e Engenharia da Computação do Centro de informática da UFPE. Atualmente, no ano de 2019, essa disciplina é ministrada pelo professor Robson do Nascimento Fidalgo (Engenharia da Computação) e pela professora Valéria Times (Ciência da Computação) e possui uma carga horária total de 75 horas. Além disso, possui como pré-requisito a disciplina de Algoritmos e Estrutura de Dados.\cite{sitedadisciplina}

\section{Relevância}
A disciplina de Gerenciamento de Dados e Informação tem por objetivo fornecer aos alunos uma base sólida em banco dados, abrangendo desde conceitos de modelagem, até recursos de linguagens de 4ª geração.\cite{if685}

\begin{figure}[h!]
\centering
\includegraphics[scale=0.3]{cientista.jpg}
\caption{Gerenciamento de Dados \cite{gerenciamentodedados}}
\label{fig:Gerenciamento de Dados}
\end{figure}

\section{Relação com outras disciplinas}
Gerenciamento de Dados e Informação possui como pré-requisito a disciplina de Algoritmos e Estrutura de Dados, que junto com a disciplina de Introdução a Programação, servem como alicerce conceitual para essa matéria. Isso ocorre visto que a mesma aborda conceitos de modelagem de dados por meio de operadores que permitem acessar e armazenar dados em uma memória principal, manipulando e atualizando os tipos de objetos, sendo muito utilizado para isso, o uso de pilhas recursivas, listas encandeadas, arrays e conceitos básicos de algoritmos e estruturas de dados.





\bibliographystyle{plain}
\bibliography{cfpj}
\end{document}
