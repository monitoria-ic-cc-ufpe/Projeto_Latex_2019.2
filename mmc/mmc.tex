\documentclass{article}
\usepackage[utf8]{inputenc}

\title{IF803 - Introdução à Biologia Molecular Computacional}
\author{Mateus Marques Coutinho - mmc }
\date{Novembro 2019}

\usepackage{natbib}
\usepackage{graphicx}

\begin{document}

\maketitle

\section{Introdução}
O objetivo deste curso é apresentar a área de Biologia Molecular Computacional, introduzindo conceitos essencias da Biologia para a compreensão da área de aplicação, dos problemas práticos de Bio-Informática e Biologia Computacional que envolve a manipulação e análise de dados biológicos e problemas da área, juntamente com abordagens computacionais para a sua solução.
Esta é uma disciplina eletiva que faz parte do perfil Bio-informática da graduação em Ciência da Computação, e está sendo oferecida em conjunto com a disciplina básica da pós-graduação em Ciência da Computação IN1115 (Introdução à Bio-informática e Biologia Computacional). \cite{Intro}

\begin{figure}[h!]
\centering
\includegraphics[scale=0.4]{Bioinformatica[2].png}
\caption{Imagem ilustrativa representando a biologia molecular computacional.} \cite{Img1}
\label{fig:Bioinf}
\end{figure}

\section{Relevância}
É fato que a união entre biologia e informática é de suma importância para o avanço e desenvolvimento de pesquisas, especialmente na área genética. Essa colaboração nasceu com a implementação do Projeto Genoma, de 1990, que tinha como objetivo decodificar e mapear totalmente o código genético humano, tarefa que seria praticamente impossível de ser realizada sem o auxílio de uma máquina, dado que a sequência genética de um ser humano possui algo na ordem de 3 • 10 elevado a 9 nucelotídeos (caracteres em {A,C,G,T}). Desde então, essa área de pesquisa só tendeu a crescer, abrangindo, a partir do fim do Projeto, em 2003, as áreas de predição de estruturas de proteínas, reconstrução de redes de genes, interação entre proteínas e genes, SNPs (single nucleotide polimorphism) e Haplotyping N. Por fim, conclue-se que as pesquisas envolvendo o código genético com o auxílio da informática são a chave para desvendarmos as informações presentes nos genes e como elas interferem na fisiologia do ser humano, tornando possível desde a cura e estudo de doenças genéticas até o milagre da clonagem, como visto na experiência da Ovelha Dolly, na década de 90. \cite{Relevancia}

\section{Relação com outras disciplinas}
Por ser uma disciplina quase que inteiramente ligada à biologia, não possui pré-requisitos diretos de outras cadeiras do curso. No entanto, por lidar intrínsecamente com a área de informática, é recomendado que o aluno domine as seguintes cadeiras:
\begin{itemize}
\item INTRODUÇÃO À PROGRAMAÇÃO (IF669): Disciplina do primeiro período, ensina os conceitos básicos de lógica de programação;
\end{itemize}
\begin{itemize}
\item ALGORTIMOS E ESTRUTURAS DE DADOS (IF672): Disciplina do segundo período, se foca na construção de programas e algoritmos cada vez mais eficientes.
\end{itemize}
Além disso, é preferível que o estudante também esteja capacitado a aplicar esses conceitos nas linguagens C, C++, Python ou Perl, preferencialmente. \cite{Relacao} 
\bibliographystyle{unsrt}
\bibliography{mmc}
\end{document}
