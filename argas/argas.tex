\documentclass[10pt]{article}
\usepackage[utf8]{inputenc}
\usepackage{ragged2e}

\title{Paradigmas de Linguagens Computacionais}
\author{Alyson Renan}
\date{Novembro, 2019}

\usepackage{natbib}
\usepackage{graphicx}

\begin{document}

\maketitle

\section{Introdução}
\justify
Paradigmas de Linguagens Computacionais é uma das disciplinas de caráter obrigatório presente na grade curricular dos cursos de Ciência da Computação\cite{ccClassPage} e Engenharia da Computação\cite{ecClassPage} do Centro de Informática da UFPE. Ela é ministrada no 5º período da graduação pelo professor Dr. Márcio Cornélio \cite{cornelioLattes}, para os alunos de Ciência da Computação, e pelos professores Dr. Andre Santos \cite{santosLattes} e Dr. Henrique Rebêlo \cite{rebeloLattes}, para os alunos de Engenharia da Computação, e tem por objetivo principal apresentar paradigmas alternativos ao imperativo.

\justify
Em sua ementa, tomam-se como principais tópicos a abordagem do paradigma funcional, do paradigma concorrente e a linguagem de script. Assim sendo, a disciplina tem por missão propor ferramentas aos alunos para que estes desenvolvam uma melhor compreensão sobre as construções utilizadas nas linguagens de programação modernas, além de fomentá-los uma visão crítica, por meio da qual eles serão capazes de definir quais linguagens utilizar para solucionar os problemas abordados de forma mais otimizada.

\section{Relevância}
\justify
Existe uma grande quantidade de linguagens de programação disponíveis no mercado atualmente, todavia nenhuma delas é ideal para a resolução de todos os tipos de problemas \cite{castroClass}. Dessa forma, surge a necessidade de conhecer as especificidades de cada uma delas e desenvolver a habilidade de selecionar quais linguagens são mais eficientes para cada contexto. Daí vem a importância do estudo dos paradigmas de linguagens de programação.

\justify
É importante que o profissional de computação disponha de um vasto arsenal de ferramentas para resolver problemas. A disciplina de paradigmas de linguagens de programação consta na grade curricular de ambos os cursos supramencionados justamente para suprir tal necessidade e, embora não vise ensinar um tipo de linguagem específica, o curso apresenta um ênfase particular na linguagem Haskell como exemplo de linguagem relacionada ao paradigma funcional, e em Java como exemplo de paradigma concorrente.

\section{Relação com outras disciplinas}
\justify
De acordo com o perfil curricular do curso de Ciência da Computação, a disciplina de Paradigmas de Linguagens de Programação não possui e também não é pré-requisito de nenhuma outra disciplina, o que não a relaciona de forma direta com as outras cadeiras do bacharelado. Entretanto, ao observarmos o perfil do curso de Engenharia da Computação, tabela \ref{tab:relacaoDisciplinas}, podemos notar que ela está diretamente relacionada com a disciplina de Introdução a Programação, o que é coerente pois, no primeiro período, o aluno é imerso em uma nova linguagem e ganha as ferramentas iniciais para a resolução de problemas e então, alguns períodos depois, são apresentadas novas ferramentas para solucioná-los, demonstrando que não há somente uma única maneira para isso.

\begin{table}[h]
\centering
\begin{tabular}{p{3.5cm}|c|p{2.5cm}|l}
Nome da Disciplina & Código & Pré-requisito & Código\\
\hline\hline
Paradigmas de Linguagens Computacionais & IF686 & Introdução à Computação & IF669
\end{tabular}
\caption{Relação entre as disciplinas de Engenharia da Computação}
\label{tab:relacaoDisciplinas}
\end{table}

\bibliographystyle{plain}
\bibliography{argas}
\end{document}