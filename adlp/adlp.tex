\documentclass{article}
\usepackage[utf8]{inputenc}

\title{IF674 - Arquitetura de Computadores}
\author{Anthonn Dayvson Lino Paz}
\date{November 2019}

\usepackage{natbib}
\usepackage{graphicx}

\begin{document}

\maketitle

\section{Introducão}
Essa área incomum abraça a inovação com uma velocidade surpreendente. Nos últimos anos, surgiram inúmeros novos computadores que prometiam revolucionar a indústria da computação; essas revoluções foram interrompidas porque alguém sempre construía um computador ainda melhor.
Essa corrida para inovar levou a um progresso sem precendentes desde o início da computação eletrônica no final da década de 1940. Os computadores levaram a humanidade a uma terceira revolução, a revolução da informação. E essa revolução ainda continua, cada vez que o custo da computação melhora por um fator de X, as oportunidades para os computadores se multiplicam.
 \citep{org}



\begin{figure}[h!]
\centering
\includegraphics[scale=0.3]{adlp-523x454.jpg}
\caption{Organização Básica de Computadores\citep{img}}
\label{fig:adlp-523x454.jpg}
\end{figure}

\section{Relevância}
Tornou novas aplicações possíveis
    Computadores em automóveis: Reduzindo à  poluição, melhorando a eficiência de combustível através  de  controles no motor, aumentou a segurança através de prevenção de situações perigosas.
    Ex : O projeto Volvo Drive me.
    
    Celulares : O que era isso a 30  anos ? Quem sonharia que os avanços dos sistemas computacionais levariam aos telefones portáteis, permitindo a comunicação pessoa a pessoa em quase todo lugar do mundo ?
    
    Projeto do genoma humano: O custo de uma màquina de sequenciamento é cerca de dezenas de milhôes de dolares, a alguns anos atrás era de cerca de 10 a 100 vezes esse valor.
    
    Wolrd Wide Web: Transformou nossa socieade.
     \citep{org}
   
  \section{Relação com outras discplinas}
  A cadeira de Infra-estrutura de Hardware se relaciona com diversas outras cadeiras, mas as quais se relaciona diretamente são: Sistemas Digitais e Infra-Estrutura de Software. Na primeira aprendemos o funciomante e como criar Sistemas/Circuitos digitais, e na segunda como é o funcionamento de um SO.
    
    
\section{Conclusão}
 Claramente, os avanços dessa tecnologia hoje afetam quase todos os aspectos da nossa sociedade. Permitindo assim que os programadores criassem softwares maravilhosamente úteis e explicassem porque os computadores nos dias de hoje são onipresentes. A ficção científica de hoje sugere as aplicações que fazem sucesso amanhã. “The science and art of designing, selecting, and interconnecting
hardware components and designing the hardware/software interface
to create a computing system that meets functional, performance,
energy consumption, cost, and other specific goals.”
 \citep{profonurmutulo}
 
 


\bibliographystyle{plain}
\bibliography{references}
\end{document}
