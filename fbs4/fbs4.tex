\documentclass[a4paper, 10pt]{article}
\usepackage[utf8]{inputenc}
\usepackage{natbib}
\usepackage{graphicx}
\usepackage[portuguese]{babel}



\title{IF743 - Segurança de sistemas}
\author{Frederico Bresani Santos}
\date{Outubro 2019}

\begin{document}
\maketitle

\section{Introdução}
Primeiramente, é necessário falar sobre Sistemas distribuídos que seria uma junção de computadores autônomos conectados por uma rede de computadores e equipados com software distribuído; os computadores se comunicam e coordenam suas atividades por troca de mensagens. Por conseguinte, alguns exemplos de sistemas distribuídos seriam o Google, Facebook, Instagram e Youtube.
Logo, a segurança de sistemas é a capacidade que um sistema tem de garantir que somente usuários autorizados tenham acesso às informações, assegurar informações confiáveis, sólidas e também garantir a disponibilidade de recursos e serviços; assim como, recuperação de senhas e de dados variados, ou seja, um suporte eficiente. Contudo, atingir uma boa qualidade da segurança de sistemas é uma tarefa muito trabalhosa e complexa a partir do momento que nos dias de hoje milhões de pessoas utilizam sistemas distribuídos para trocar diversos tipos de informação.
Seguidamente, é possível observar que muitas empresas e até bancos usam esses sistemas distribuídos, para realizar transferências, fechar negócios e efetuar pagamentos. Logo, se algum ataque ou invasão desses sistemas ocorrerem prejuízos financeiros aconteceriam; o que prejudicaria a economia e a imagem das empresas. Devido a isso, é aferível que a segurança de sistemas se tornou algo muito valioso para a sociedade atual.
\citep{enterocistoma}



\begin{figure}[h!]
\centering
\includegraphics[scale=0.9]{ss.jpg}
\caption{Sistemas de segurança \citep{imagemSeguranca}}
\label{fig:ss}
\end{figure}



\section{Relevância}
A segurança de sistemas se relaciona com a maioria das atividades computacionais já que atualmente o mundo está conectado pela internet. Devido a isso, é preciso muitas pesquisas e desenvolvimento nessa área. Um método de segurança muito importante se chama a criptografia. A mesma, remete a prática da codificação e decodificação de dados. Por conseguinte, a criptografia consiste em aplicar um algoritmo a um certo dado de uma maneira que eles não tenham mas seu formato original e não possam ser entendidos; esse dado só pode ser decodificado se for aplicada uma chave específica. 
Esse trabalho constitui uma parte importante da segurança de dados e protegem diversos tipos de informação de “ataques” ou invasões de terceiros.
\citep{criptografia}

\begin{figure}[h!]
\centering
\includegraphics[scale=0.23]{cripto.jpg}
\caption{Criptografia \citep{imagemcripto}}
\label{Criptografia}
\end{figure}

\section{Relação com outras cadeiras}
A segurança de sistemas se relaciona com a cadeira de criptografia (ES - 268), já que a criptografia é fundamental para o desenvolvimento da segurança de diversos tipos de dados.



\bibliographystyle{plain}
\bibliography{fbs4}
\end{document}
