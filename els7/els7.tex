\documentclass{article}
\usepackage[utf8]{inputenc}

\title{Lógica para computação}
\author{Edvyn Luiz}
\date{Novembro 2019}

\usepackage{natbib}
\usepackage{graphicx}

\begin{document}

\maketitle

\section{Introdução}
O curso de lógica para computação é uma introdução às técnicas de raciocínio dedutivo,usando ferramentas da lógica matemática. Esta,por sua vez, estuda as noções de validade e consistência de argumentos fazendo o uso de elementos matemáticos,tais como a teoria dos conjuntos e álgebra booleana. A lógica para computação se encontra presente nas sub-áreas de matemática computacional e computação básica se fazendo de total importância para os cursos de tecnologia da informação.
\citep{cinufpe2019}

\begin{figure}[h!]
\centering
\includegraphics[scale=1]{logica.png}
\caption{Pensamento lógico}
{Fonte: metodosupera.com.br}
\label{fig:logica}
\end{figure}
Os fundamentos mais essenciais para a ciência da computação são baseados em lógica e teoria dos conjuntos. O lógico Gottlob Frege que definiu o primeiro cálculo proposicional essencialmente criou a primeira linguagem de programação. A linguagem que ele definiu tem todos os requisitos formais para uma poderosa linguagem de programação e especificação de computadores. A teoria da computação se baseia em conceitos definidos pelos lógicos e matemáticos, como Alonzo Church e Alan Turing.
\cite{wikipedia2019}

\section{Relevância}
A presença de lógica para computação nos componentes curriculares do curso se faz pelo fato de que os fundamentos mais essenciais para a ciência da computação são baseados em lógica e teoria dos conjuntos.Ao ingressar no curso de lógica para computação,os alunos podem sentir algum tipo de dificuldade ao se depararem com os problemas propostos,por outro lado,ao terminarem o curso estarão com uma boa base para prosseguir em sua carreira acadêmica.
\citep{ufpe2019}
\citep{cinufpe2019}

\section{Relação com outras disciplinas}

No curso de Ciência da Computação,a disciplina de lógica tem como pré-requisito Matemática Discreta,onde você tem uma introdução ao mundo da lógica ao estudar teoria dos conjuntos. E por sua vez,a lógica é pré-requisito de Engenharia de software,sistemas inteligentes e informática teórica.Além de se relacionar com programação, por estudar o raciocínio necessário para a construção de algoritmos.
\citep{ufpe2019}
\bibliographystyle{plain}
\bibliography{els7}
\end{document}
