\documentclass{article}
\usepackage[utf8]{inputenc}
\usepackage{natbib}
\usepackage{graphicx}

\title{IF689 - Informática Teórica}
\author{Júlio Vinícius Gonçalves dos Santos }
\date{November, 2019}

\begin{document}

\maketitle

\section{Introdução}
\quad Informática teoria é uma cadeira obrigatória do 4º período que aborda conceitos da teoria da computação, nessa disciplina é abordado assuntos como Definição de Algoritmos, Autômatos Finitos, Expressões regulares, Teoria da complexidade, podemos destacar o estudo das máquinas de Turing \citep{cinWiki} , a disciplina é lecionada pelo Prof. Ruy de Queiroz ou Prof. Fred Freitas.\citep{cinWiki2}

\begin{figure}[h!]
\centering
\includegraphics[scale=0.22]{Mturing}
\caption{máquina de turing. \cite{imagem}}
\end{figure}


\section{Relevância}
\quad Essa disciplina mostra conceitos teóricos importantes utilizados em outras disciplinas e áreas da computação, com isso os alunos terão base para inúmeras aplicações práticas da computação, como criptografia, uso de autômatos finitos, verificações de programas, entre outros, conceitos usados em outras cadeiras e no mercado, essa cadeira também faz parte do curso de engenharia da computação.
\section{Relação com outras disciplinas}
\quad A disciplina de Informática teorica possui como pré-requisito as disciplinas de Algoritmos e Estruturas de Dados e Lógica para computação e é pré-requisito de várias disciplinas, podemos destacar teoria da prova, teoria da recursão, teoria de grafos, teoria de modelos. \citep{ufpe}
\bibliographystyle{siam}
\bibliography{jvgs}
\end{document}
