\documentclass{article}
\usepackage[utf8]{inputenc}
\usepackage[brazil]{babel}
\title{IF687 -Introdução à Multimídia}
\author{Gabriel Henrique}
\date{Novembro 2019}

\usepackage{natbib}
\usepackage{graphicx}

\begin{document}

\maketitle

\section{Introdução}
A disciplina de Introdução a Multimídia tem como foco desenvolver o estudo que é referente as percepções sensoriais, criando novas perspectivas a determinados temas, para torná-los mais entendíveis. Logo, aplicar esses conhecimentos em diversas áreas. Por fim, alguns tópicos desta disciplina são: 
\begin{itemize}
    \item Realidade Virtual: Que tem como objetivo criar novos ambientes independentes da realidade, e junto a isso, causar uma sensação de presença dentro deste âmbito virtual. 
    \item Realidade Aumentada: Que é responsável por criar à interação com a realidade através das mídias, um exemplo bem popular de realidade aumentado é o jogo Pokémon GO.
    \item Processamento Gráfico: Que atua na importante área visual da multimídia.
\end{itemize}

\begin{figure}[h!]
\centering
\includegraphics[width=100mm]{virtua.jpg}
\caption{Exemplo de Realidade Virtual}
\label{fig:Realidade Virtual}
\end{figure}

\section{Relevância}
A relevância da Introdução a multimídia é indubitável, pois é através dessa disciplina que existe a interação dos estudantes com um novo universo virtual. Sabendo disso, é possível trabalhar com objetos virtuais e adicionar estes ao mundo real, afim de principalmente beneficiar nas suas aplicações, sendo a multimídia amplamente explorada nas áreas de educação, entretenimento e medicina, por exemplo. Deste modo, é extremamente relevante o estudo desta disciplina, visto que está bastante comprovada a sua importância para a formação de novos profissionais da computação.

\section{Relação com outras disciplinas}


\vspace{0.3cm}
\begin{tabular}{|p{3.0cm}|p{7.0cm}|}
\hline
Disciplinas & Relação\\
\hline


IF669 - Introdução â programação & É fundamental que se tenha conhecimento prévio sobre programação, pois ensina a manipular dados virtuais\\
\hline 
MA531 - Álgebra Vetorial Linear para Computação & Proporciona conhecimentos essenciais para a manipulação de dados visuais \\
\hline 

IF755 - Realidade Virtual & Ensino mais completo e aprofundado da Introdução a Multimídia\\
\hline
IF680 - Processamento Gráfico & A disciplina de processamento gráfico tem o foco de estudar assuntos diretamente relacionados ao visual, fundamentais para o entendimento dos âmbitos virtuais \\
\hline 



\end{tabular}

\bibliographystyle{plain}
\bibliography{ghrp}
\nocite{*}
\end{document}